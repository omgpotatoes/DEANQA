
%==================================
% packages go here
%==================================

\usepackage{url}
\usepackage{graphicx}
\usepackage{cite} % sort citation numbers
\usepackage{color}
\usepackage{soul}   
%\usepackage{ntheorem}
%\usepackage{amsthm}
%\usepackage{multirow} % for complex tables
%\usepackage{times}
\usepackage[pdftex]{hyperref}
\hypersetup{colorlinks=false, pdfborder= 0 0 0}
\usepackage[draft,inline,nomargin]{fixme}
\newcommand{\hlfixme}[1]{\fixme{\hl{#1}}}
\newcommand{\hlfxnote}[1]{\fxnote{\hl{#1}}}

%==================================

\documentclass{acm_proc_article-sp}

\begin{document}

\title{Watson Jr.}
\subtitle{[A Question Answering System]}

\numberofauthors{3}
\author{
\alignauthor Name 1 \\
       \affaddr{University of Pittsburgh}\\
       \affaddr{Department of Computer Science}\\
       \email{email@domain.edu}
% author
\alignauthor Yann Le Gall\\
       \affaddr{University of Pittsburgh}\\
       \affaddr{Department of Computer Science}\\
       \email{ylegall@cs.pitt.edu}
% author
\alignauthor Name 3 \\
       \affaddr{University of Pittsburgh}\\
       \affaddr{Department of Computer Science}\\
       \email{email@domain.edu}
}

\date{12 December 2011}
\maketitle

%==================================

\begin{abstract}
abstract goes here
\end{abstract}

% A category with the (minimum) three required fields
\category{I.2}{Artifical Intelligence}{Natural Language Processing}
\terms{Algorithms, Experimentation}
\keywords{Question Answering}

%==================================

\section{Introduction}
\lablel{sec:intro}

In this paper, we design, implement, and evaluate a question answering
(QA) system. In our approach, we use several different strategies from
different domains to select potential answers. Then, we employ a
majority voting scheme to combine the results.

% TODO: should this description of the dataset go in another section?
To train and test our QA system, we used the ``CBC Reading
Comprehension Corpus''. This corpus is composed of 125 news stories,
each accompanied by a set of 6-10 factoid questions (e.g. questions
that begin with ``Who'', ``When'', ``Where'', etc.).
The news stories were obtained from the ``CBC 4 Kids'' website,
hosted by the Canadian Broadcast Corporation. The questions and an
answer key were added by the MITRE Corporation, and are in the style
of actual reading comprehension tests that are given to grade school
children in the United States.

The rest of this paper is organized as follows: in
\S\ref{sec:implementation} we describe the design and implementation
of our QA system and each of its sub-components. Next, in
\S\ref{sec:evaluation} we evaluate our system on the test dataset and
present the performance results. Finally, we conclude in
\S\ref{sec:conclusion}.


\section{Implementation}
\lablel{sec:implementation}

In this section we discuss the design and implementation of our QA
system. First, we explain the general, overall structure. Then we give
a detailed description of the various sub-components, each of which
implements a different strategy to answer questions. Finally, we
present our technique for combining the potential answers from each
sub-component.

%TODO: show a diagram of the system

\subsection{Naive Bag-of-Words}

% TODO: Yann writes this

\subsection{Linguistic Rule-based Strategies}

% TODO: Yann writes this

\subsection{NLP Components}

% TODO: Alex writes this

\subsection{SVM Classification}

% TODO: Eric writes this part

\subsection{Combining Answers with Majority Voting}
Previous work by Rotaru and Litman demonstrates that combining
the outputs of multiple QA systems can achieve better results than the
individual systems alone \cite{rotaru2005}. We incorporate this idea
into the design of our QA system by combining the outputs from each of
the subsystems described above.

% TODO: describe how we combine the different results

\section{Evaluation}
\lablel{sec:evaluation}



\section{Conclusions}
\lablel{sec:conclusion}

Question-answering is complex task and a cutting-edge research area
in natural language processing. 


%ACKNOWLEDGMENTS are optional
%\section{Acknowledgments}

%\bibliographystyle{abbrv}
\bibliographystyle{plain}
\bibliography{references} 

%
%\appendix
%%Appendix A
%\section{Headings in Appendices}
%\balancecolumns


\end{document}

